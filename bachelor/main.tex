% В этом файле следует писать текст работы, разбивая его на
% разделы (section), подразделы (subsection) и, если нужно,
% главы (chapter).

% Предварительно следует указать необходимую информацию
% в файле SETUP.tex

\input{preamble.tex}
\graphicspath{ {img/} }

\NewBibliographyString{langjapanese}
\NewBibliographyString{fromjapanese}

\begin{document}

\Intro
В данной работе была реализована генерация алгебраических типов данных языка программирования \lstinline{Haskell} 
по исходным данным в формате \lstinline{JSON}.

В качестве основных инструментов для реализации поставленной задачи используются 
библиотека \lstinline{Data.Aeson}~\cite{aeson} и расширение \lstinline{Template Haskell}~\cite{tempHaskell}. Генерация 
производится с помощью рекурсивного прохода по абстрактному синтаксическому дереву (далее --- AST) и 
накопление результатов обхода в монаде \lstinline{State} ~\cite{stateM}.

Ключевыми моментами в реализации программы являются:

\begin{itemize}
  \item Получение AST по исходному \lstinline{JSON}.
  \item Преобразование AST в структуру для генерации алгебраического типа.
  \item Вклейка («splicing») сгенерированного типа данных в код.
\end{itemize}

\chapter{Предварительные сведения}

\section{JavaScript Object Notation}

JSON (\lstinline{JavaScript Object Notation})~\cite{jsonStandart} --- простой формат обмена данными, основанный на подмножестве языка программирования \lstinline{JavaScript}. При этом рассматривемый формат независим от реализации и может использоваться любым языком программирования. Файл в формате \lstinline{JSON} представляет собой неупорядоченное множество пар ключ-значение, значения которого могут иметь следующий тип:  

\begin{itemize}
  \item объекты (выделяются \{ ... \}),
  \item массивы (выделяются [ ... ]),
  \item строки,
  \item логические выражения (true | false),
  \item null-значения.
\end{itemize}

\begin{ListingEnv}[H]
\begin{Verb}
{     
    "firstName": "John",
    "lastName" : "Smith",
    "age" : 25
}
\end{Verb}
\caption{Пример данных в формате JSON}
\label{listing:jsonExample}
\end{ListingEnv}

Наглядно продемонстрировать вышеобозначенную структуру можно при помощи схем, представленных на рисунках ~\ref{fig:objectGr}, ~\ref{fig:arrayGr} и ~\ref{fig:valueGr}.

\begin{figure}[!ht]
\centering
\includegraphics[width=\textwidth]{object}
\caption{\label{fig:objectGr}Объект}
\end{figure}

\begin{figure}[!ht]
\centering
\includegraphics[width=\textwidth]{array}
\caption{\label{fig:arrayGr}Массив}
\end{figure}

\begin{figure}[!ht]
\centering
\includegraphics[width=\textwidth]{value}
\caption{\label{fig:valueGr}Значение}
\end{figure}

\section{Алгебраические типы данных}

Алгебраические типы данных (далее --- АТД) --- вид составных типов, представленных типом-произведением, типом-суммой, либо комбинацией: суммой произведений.~\cite{haskellGreatGood} Последний вариант можно проиллюстрировать на примере двоичного дерева:

\begin{lstlisting}[language=Haskell]
data Tree a = Leaf a
          | Node (Tree a) (Tree a)
\end{lstlisting}

В примере \lstinline{Tree} является конструктором типа с одним типовым параметром. \lstinline{Leaf} и \lstinline{Node} --- конструкторы значений. 

Также существует другой вариант определения типа, который называется синтаксисом записи с именованными полями:

\begin{lstlisting}[language=Haskell]
data Person = Person { firstName :: String
                       , lastName :: String
                       , age :: Int }
\end{lstlisting}

Вместо простого перечисления типов, мы создаем некую структуру и наполняем ее полями и их значениями. Главное преимущество заключается в том, что подобный синтаксис генерирует функции для извлечения полей. Также облегчается чтение и понимание типа данных. 

Стоит отметить, что с помощью АТД и стандартных типов \lstinline{Haskell} (например, строковых String, списковых [a] и т. д.) можно представить \lstinline{JSON}-структуру.
\section{Монада State}

Монада \lstinline{State} применима в тех случаях, когда имеется некоторое состояние, которое подвергается постоянным изменениям. Cтоит отметить, что при таком способе мы трансформируем состояние, но при этом не теряем "чистоту" \: функций. 

\subsection{Control.Monad.State}

Модуль Control.Monad.State~\cite{stateControl} определяет тип, оборачивающий вычисление с состоянием:

\begin{lstlisting}[language=Haskell]
newtype State s a = State {runState :: s -> (a,s) }
\end{lstlisting} 

Как следует из определения, вычисление в монаде State возвращает некоторый результат и при этом в случае необходимости меняет состояние. Операции с состояниями реализованы следующими функциями: \lstinline{get}\:(получает состояние) и \lstinline{put}\:(изменяет состояние на заданное). В листинге~\ref{listing:stateGetPut} приведен простой пример, позволяющий продемонстрировать возможности монады \lstinline{State}.

\begin{ListingEnv}[H]
\begin{Verb}
tick :: State Int Int
tick = do n <- get
	  put (n+1)
          return n

ghci> runState tick 3
(3,4)
\end{Verb}
\caption{Пример использования монады State}
\label{listing:stateGetPut}
\end{ListingEnv}

Также модуль содержит и другие полезные функции для работы с состояниями. Самые важные из них представлены в листинге~\ref{listing:stateFunc}. Функции с постфиксом -State отличаются типом возвращаемого значения. Необходимое поведение можно выбрать исходя из сигнатуры.

\begin{ListingEnv}[H]
\begin{Verb}
modify :: MonadState s m => (s -> s) -> m ()

execState:: State s a -> s -> s

runState:: State s a -> s -> (a, s)

evalState:: State s a -> s -> a
\end{Verb}
\caption{Функции модуля Control.Monad.State}
\label{listing:stateFunc}
\end{ListingEnv}

Например, функция \lstinline{modify} преобразует внутреннее состояние функцией, которую получает на вход. Можно реализовать код листинга ~\ref{listing:stateGetPut} через \lstinline{modify} (листинг~\ref{listing:modifyState}).

\begin{ListingEnv}[H]
\begin{Verb}
tick :: State Int Int
tick = do modify (+1)
          return n

ghci> runState tick 3
(3,4)
\end{Verb}
\caption{Функция modify модуля Control.Monad.State}
\label{listing:modifyState}
\end{ListingEnv}
	
\section{Библиотека Data.Aeson}
\label{sec:secAeson}

\subsection{Обзор}

\lstinline{Data.Aeson} --- библиотека для работы с файлами в формате \lstinline{JSON}, написанная на языке \lstinline{Haskell}. В данной  библиотеке используются два основных класса типов --- \lstinline{FromJSON} и \lstinline{ToJSON}.~\cite{aesonEx} Типы, имеющие возможность кодирования/декодирования, должны быть экземплярами классов \lstinline{FromJSON}, \lstinline{ToJSON}. Самый простой способ использования библиотеки заключается в определении типов данных и экземпляров \lstinline{FromJSON}, \lstinline{ToJSON}. 

Существует возможность прописать экземпляры для кодирования/декодирования по умолчанию благодаря инструкции компилятора \lstinline{LANGUAGE} и экземпляру \lstinline{Generic}. Рассмотрим листинг~\ref{listing:genericData}, демонстрирующий данную возможность с условным типом данных. 

\begin{ListingEnv}[H]
\begin{Verb}
instance ToJSON DataName

instance FromJSON DataName
\end{Verb}
\caption{Создание экземпляров по умолчанию}
\label{listing:genericData}
\end{ListingEnv}

\lstinline{Data.Aeson} имеет свой собственный тип для представления конвертируемого \lstinline{JSON}-файла. Этот тип называется \lstinline{Value} и имеет шесть конструкторов значения:

\begin{ListingEnv}[H]
\begin{Verb}
data Value
  = Object Object
  | Array Array
  | String Text
  | Number Scientific
  | Bool Bool
  | Null
\end{Verb}
\caption{Конструкторы Value}
\label{listing:value}
\end{ListingEnv}

\subsection{Работа с AST}

\lstinline{Aeson} позволяет получить абстрактное синтаксическое дерево по \lstinline{JSON}. Это бывает полезно в случаях, когда неизвестен тип данных, соответствующий входному файлу. Имея AST, можно написать функцию для его обхода.~\cite{aesonEx}

В качестве примера рассмотрим получение AST для двух случаев: простой (листинг~\ref{listing:astGetSimple}) и более сложный со вложенными объектами (листинг~\ref{listing:astGetComp}).

\begin{ListingEnv}[H]
\begin{Verb}
decode :: FromJSON a => ByteString -> Maybe a

ghci> decode "{\"foo\": 123}" :: Maybe Value
Just (Object (fromList [("foo",Number 123)]))
\end{Verb}
\caption{JSON без вложенных объектов}
\label{listing:astGetSimple}
\end{ListingEnv}

\begin{ListingEnv}[H]
\begin{Verb}
ghci> decode "{\"foo\": [\"abc\",\"def\"]}" :: Maybe Value
Just (Object (fromList [("foo",Array (fromList [String "abc", 
                                               String "def"]))]))
\end{Verb}
\caption{JSON со вложенными объектами}
\label{listing:astGetComp}
\end{ListingEnv}

\section{Template Haskell}

\lstinline{Template Haskell} --- это расширение языка \lstinline{Haskell}, реализующее средства для метапрограммирования.~\cite{extensionHub} Оно позволяет использовать Haskell как язык управления и как управляемый язык. Поскольку работа ведется с расширением, в исходник необходимо добавить директиву:

\begin{lstlisting}[language=Haskell]
{-# LANGUAGE TemplateHaskell #-}.
\end{lstlisting}

\subsection{Монада Q}

Монада \lstinline{Q} оборачивает значения типов, предназначенных для последующей генерации других конструкций на языке \lstinline{Haskell}. Такие типы полностью удовлетворяют синтаксису языка и представляют собой абстрактное синтаксическое дерево кода на \lstinline{Haskell}: тип \lstinline{Exp} --- для генерации выражений, \lstinline{Dec} --- для объявлений, \lstinline{Lit} --- для литералов и т.д.~\cite{thSyntax}

\subsection{Модуль Language.Haskell.TH.Syntax}

Тип \lstinline{Exp} определен в модуле \lstinline{Language.Haskell.TH.Syntax}~\cite{coverHaskell} и  \lstinline{Exp} имеет семнадцать конструкторов значения. В листинге ~\ref{listing:expConstr} представлены первые четыре из них.  

\begin{ListingEnv}[H]
\begin{Verb}
data Exp
       = VarE Name
       | AppE Exp Exp
       | MultiIfE [(Guard, Exp)]
       | CondE Exp Exp Exp
       | ...
\end{Verb}
\caption{Конструкторы Exp}
\label{listing:expConstr}
\end{ListingEnv}

\subsection{Вклейка (splicing)}

Вклейка кода производится оператором  \lstinline{\$(...)}, который разворачивает шаблон с данным параметром в обычный \lstinline{Haskell}-код во время компиляции и вклеивает его на то же место. Важно, чтобы между скобками и оператором \lstinline{\$} не было пробелов. Также существует ограничение по использованию вклейки: ее можно применить только из другого модуля, т.е. отсутствует возможность "вклейки" шаблона в одном модуле вместе с его определением.

Вклейка может использоваться в:

\begin{itemize}
  \item выражениях (выражение должно иметь тип Q Exp);
  \item объявлениях верхнего уровня (выражение должно иметь тип Q [Dec]);
  \item типах (выражение должно иметь тип Q Type).
\end{itemize}

\chapter{Генерация АТД по JSON-файлу}

\section{Выбор средств для реализации}

Изначально при реализации задачи, поставленной во введении, стоял выбор между несколькими средствами. Рассмотрим каждое из них.

\subsection{Расширение к языку Haskell}

При подобной реализации становятся очевидными некоторые преимущества, связанные с простотой использования (необходимо лишь прописать директиву, добавляющую это расширение), но при этом она представляется сложной в реализации, поскольку для ее осуществления необходимо изменить исходные коды компилятора  \lstinline{GHC}. При этом в данной реализации не хватало бы наглядности и синтаксис отличался бы от привычного.

\subsection{Написание парсера}
Данное решение позволяет использовать стандартные методы и наработки, связанные с парсерами. При этом можно вывести готовый алгебраический тип данных. Но в данном случае возникает проблема: как использовать этот тип незамедлительно без дополнительных манипуляций.   	

\subsection{Template Haskell}
Данный подход оказался предпочтительнее остальных по ряду причин. Во-первых, при его использовании отсутствуют недостатки, присущие другим подходам, в том числе существует возможность использования сгенерированного типа без дополнительных накладных ресурсов и средств. Во-вторых, предоставляется возможность генерирования данных  и последующего незамедлительного их использования. Для получение AST будет использоваться ранее упомянутая библиотека \lstinline{Data.Aeson}. Помимо этого, данная идея представляется более интересной с точки зрения практической реализации. 

\section{Программная реализация генератора АТД}

\subsection{Применение DataD (Template Haskell)}

Программная реализация средствами \lstinline{Template Haskell} генерирует объявление типов данных: используется конструктор значения \lstinline{DataD} типа \lstinline{Dec} (листинг ~\ref{listing:dataDTH}). \lstinline{DataD} является конструктором значения \lstinline{Dec}. \lstinline{Cxt} определяет классы типов. 

\lstinline{Name} --- абстрактный тип, представляющий имена в синтаксическом дереве. Используется для определения имени полей и конструкторов. Функция \lstinline{mkName} (листинг~\ref{listing:mkName}) создает значение типа \lstinline{Name} из обычной строки (\lstinline{String}), с ее содержанием в качестве имени.

\begin{ListingEnv}[H]
\begin{Verb}
Dec 
   = DataD Cxt Name [TyVarBndr] [Con] Cxt
   | ...
\end{Verb}
\caption{Конструктор значения типа Dec}
\label{listing:dataDTH}
\end{ListingEnv} 

\begin{ListingEnv}[H]
\begin{Verb}
mkName :: String -> Name
\end{Verb}
\caption{Особенность: чистота функции mkName}
\label{listing:mkName}
\end{ListingEnv} 

В коде реализации мы используем тип из листинга~\ref{listing:dataDTH} для вклейки в объявлениях верхнего уровня. \lstinline{RecC} дает понять, что сгенерированный тип должен представлять собой запись с именованными полями. Использование класса типов \lstinline{Generic} пригодится в будущем, когда будут генерироваться экземпляры классов \lstinline{FromJSON} и \lstinline{ToJSON} для вклеиваемого типа, описанные в разделе~\ref{sec:secAeson}.

\begin{ListingEnv}[H]
\begin{Verb}
DataD
     []
     (mkName $ firstLetterToUpper key')
     []
     [ RecC (mkName $ firstLetterToUpper key')  (result) ]
     [mkName "Generic", mkName "Show", mkName "Eq"]
\end{Verb}
\caption{Генерация Data в тексте программы}
\label{listing:dataDTHjson}
\end{ListingEnv} 

Важно помнить, что имена типов данных в \lstinline{Haskell} начинаются с заглавных букв. Поскольку известно, что вероятность возникновения сложных случаев высока, очевидна полезность функции \lstinline{firstLetterToUpper} (листинг~\ref{listing:letterChange}).

\begin{ListingEnv}[H]
\begin{Verb}
firstLetterToUpper :: String -> String
firstLetterToUpper fieldName = (Data.Char.toUpper 
                                         \$ DList.head fieldName)
                                         : (DList.tail fieldName)
\end{Verb}
\caption{Функция смены первой буквы на заглавную}
\label{listing:letterChange}
\end{ListingEnv} 

\section{Использование монады State}

Для аккумуляции типов \lstinline{Dec} был необходим аналог глобальной переменной в императивных языках программирования. Бесспорно, в функциональных языках отсутствуют подобные средства в чистом виде, однако похожего поведения можно добиться, используя монаду \lstinline{State}:

\begin{lstlisting}[language=Haskell]
State [Dec] ()
\end{lstlisting}

Накапливатся типы для генерации будут с помощью функции \lstinline{modify} из модуля \lstinline{Control.Monad.State}. Однако существует проблема: каждый раз, когда используется \lstinline{modify}, порождается новый экземпляр типа \lstinline{State [Dec] ()}, то есть условно при таком подходе не будет глобальной переменной, в которую собираются все необходимые данные. 

Для того чтобы решить возникшую проблему, нам необходима монадическая свертка по ключу и значению (листинг~\ref{listing:foldMonad}). Свертка реализована с помощью \lstinline{foldlM} из модуля \lstinline{Data.Foldable} и функции \lstinline{uncurry} (преобразует каррированную функцию в функцию, принимающую пару).

Еще одно преимущество использования функционального языка зак

\begin{ListingEnv}[H]
\begin{Verb}
import qualified Data.HashMap.Strict as StrHash
import qualified Data.Foldable    as FB

foldlWithKeyM :: (Monad m) => (b -> k -> a -> m b) -> b ->
                                       StrHash.HashMap k a -> m b
foldlWithKeyM f b = FB.foldlM f' b . StrHash.toList
  where f' a = uncurry (f a)
\end{Verb}
\caption{foldlWithKeyM}
\label{listing:foldMonad}
\end{ListingEnv} 

\section{Основная часть}

Основной функцией, которая запускает проход по AST и начинает накапливать значения в монаде \lstinline{State} является: 

\begin{lstlisting}[language=Haskell]
convertObject:: String -> Value -> State [Dec] ()
\end{lstlisting}

Функция \lstinline{convertObject} получает на вход имя типа данных и AST, полученное с помощью \lstinline{decode}(\lstinline{Data.Aeson}). Затем происходит спуск по дереву и анализ его вершин:

\begin{lstlisting}[language=Haskell]
convertFields:: MonadState [Dec] m => Value -> 
                                         m [(Name, Strict, Type)]
\end{lstlisting}

Как следует из сигнатуры, анализ и спуск происходит в монаде. Именно при проходе по вершинам мы используем написанный ранее \lstinline{foldlWithKeyM}. Главная задача --- учитывать, что при обходе мы можем встретить вложенные в объект объекты и вызвать функцию рекурсивно. Для этого мы используем функцию \lstinline{isObject} (листинг~\ref{listing:objectVal}). \lstinline{convertFields} получает массив из \lstinline{(Name, Strict, Type)}, который подставляется в (листинг~\ref{listing:dataDTHjson}) вместо переменной \lstinline{result}.

\begin{ListingEnv}[H]
\begin{Verb}
isObject :: Value -> Bool
isObject (Object obj) = True;
isObject _ = False;
\end{Verb}
\caption{Проверка на принадлежность Object}
\label{listing:objectVal}
\end{ListingEnv} 

Все необходимые для генерации данные собираются за один проход AST. При анализе дерева, в том случае если алгоритм доходит до листового узла, вызывается \lstinline{return}, который возвращает \lstinline{[(Name, Strict, Type)]} (листинг~\ref{listing:otherwiseTree}). И далее продолжается обход. \lstinline{MState} --- квалифицированый импорт модуля \lstinline{Control.Monad.State}.

\begin{ListingEnv}[H]
\begin{Verb}
do
      (MState.return (((mkName \$  key'), NotStrict,
                                (mkValType val' key') ) : list'))
\end{Verb}
\caption{Простой случай при обходе}
\label{listing:otherwiseTree}
\end{ListingEnv} 

Если при проходе по дереву срабатывает условие из листинга~\ref{listing:otherwiseTree}, происходит рекурсивный спуск --- вызов функции \lstinline{convertFields} с вложенным найденным объектом (листинг ~\ref{listing:isObjectTree}).

\begin{ListingEnv}[H]
\begin{Verb}
do
      result <- convertFields \$ val'
      MState.modify ((Prelude.++) [DataD ... ])
      (MState.return (((mkName \$  key'), NotStrict,
                                (mkValType val' key') ) : list'))
\end{Verb}
\caption{Сложный случай при обходе}
\label{listing:isObjectTree}
\end{ListingEnv} 

Подобным образом осуществляется вся работа программы. Обрабатывается каждый вложенный объект и каждый лист AST. На выходе получаем структуру \lstinline{[Dec]} и оборачиваем ее в монаду \lstinline{Q} для вызова вклейки из другого модуля через оператор \lstinline{\$(...)}.

\chapter{Пример практического использования}

\lstinline{Git}-репозиторий с исходным кодом на языке \lstinline{Haskell} доступен по адресу ~\cite{diploma}. Тестовые примеры запускались на компиляторе \lstinline{GHC} версии \lstinline{7.10.3}. Версия библиотеки \lstinline{Data.Aeson} --- \lstinline{0.10.0.0}, \lstinline{Template Haskell} --- \lstinline{2.10.0.0}.

\section{Отладка программ в GHC}

Компилятор \lstinline{GHC} предоставляет мощные инструменты для отладки программы. ~\cite{debugGHC} К примеру, можно ставить контрольные точки, получать подробное описание ошибок, генерировать полезные структуры, смотреть на AST и т.д. Для этого используются флаги компилятору. 

Полезным флагом для программы, использующей \lstinline{Template Haskell} служит \lstinline{-ddump-splices}. Его средствами организован вывод полученного из шаблона выражения либо ошибки (информативное сообщение).

\begin{ListingEnv}[H]
\begin{Verb}
on.hs:8:3-17: Splicing declarations
    getDataFromJSON
  ======>
    data JSONData
      = JSONData {name :: String}
      deriving (Generic, Show, Eq)
\end{Verb}
\caption{Запуск отладчика с флагом -ddump-splices}
\end{ListingEnv}

\section{Простой пример}
Для начала будет рассмотрен простой пример. На вход программе подается простой (без вложенных объектов) \lstinline{JSON} (листинг ~\ref{listing:json1}).

\begin{ListingEnv}[H]
\begin{Verb}
{
    "name" : "Joe",
    "age" : 25,
    "avg" : 4,
    "arr" : [1,2,3]
}
\end{Verb}
\label{listing:json1}
\end{ListingEnv}

В итоге мы получаем тип данных с именованными полями, полностью соответствующий постановленной задаче:

\begin{lstlisting}[language=Haskell]
data JSONData 
        = JSONData {arr :: [Float]
                    name :: String,
                    age :: Float,
                    avg :: Float}
          deriving (Show, Eq, Generic)
\end{lstlisting}

\section{Более сложный пример}

На вход программе подается \lstinline{JSON} со вложенным объектом (листинг ~\ref{listing:json2}).  Небезынтересно и то, как он будет представлен. 

\begin{ListingEnv}[H]
\begin{Verb}
{
    "name" : "Joe",
    "age" : 25,
    "avg" : 4,
    "arra" : 
             {
                 "fg" : "JSONTest"  
             }
}
\end{Verb}
\label{listing:json2}
\end{ListingEnv}

После выполнения программы мы получаем два типа данных, что полностью соответствует заявленным требованиям к программе. Первый тип в своем определении использует второй. 

\begin{lstlisting}[language=Haskell]
data JSONData
      = JSONData {name :: String,
                  arra :: Arra,
                  age :: Float,
                  avg :: Float}
        deriving (Show, Eq, Generic)

data Arra
      = Arra {fg :: String}
        deriving (Show, Eq, Generic)            
\end{lstlisting}

\Conc

В работе была рассмотрена идея метапрограммирования и поставлена задача генерации алгебраических типов данных. Также была разобрана работающая реализация задачи на функциональном языке программирования \lstinline{Haskell}. При этом были наглядно продемонстрированы результаты запусков программы в виде созданных типов данных.

При подготовке были глубоко изучены сильные стороны функционального программирования, расширение \lstinline{Template Haskell} и сопутствующие библиотеки.

Эта работа дает представление о возможностях метапрограммирования на языке \lstinline{Haskell}, а также формирует представление о многообразии задач, в которых возможно его применение.	

% Печать списка литературы (библиографии)
\printbibliography[%{}
    heading=bibintoc%
    %,title=Библиография % если хочется это слово
]
% Файл со списком литературы: biblio.bib
% Подробно по оформлению библиографии:
% см. документацию к пакету biblatex-gost
% http://ctan.mirrorcatalogs.com/macros/latex/exptl/biblatex-contrib/biblatex-gost/doc/biblatex-gost.pdf
% и огромное количество примеров там же:
% http://mirror.macomnet.net/pub/CTAN/macros/latex/contrib/biblatex-contrib/biblatex-gost/doc/biblatex-gost-examples.pdf

\end{document}
