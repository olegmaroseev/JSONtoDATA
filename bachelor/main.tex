% В этом файле следует писать текст работы, разбивая его на
% разделы (section), подразделы (subsection) и, если нужно,
% главы (chapter).

% Предварительно следует указать необходимую информацию
% в файле SETUP.tex

%% В этот файл не предполагается вносить изменения

% В этом файле следует указать информацию о себе
% и выполняемой работе.

\documentclass [fontsize=14pt, paper=a4, pagesize, DIV=calc]%
{scrartcl}
% ВНИМАНИЕ! Для использования глав поменять
% scrartcl на scrreprt

% Здесь ничего не менять
\usepackage [T2A] {fontenc}   % Кириллица в PDF файле
\usepackage [utf8] {inputenc} % Кодировка текста: utf-8
\usepackage [russian] {babel} % Переносы, лигатуры

%%%%%%%%%%%%%%%%%%%%%%%%%%%%%%%%%%%%%%%%%%%%%%%%%%%%%%%%%%%%%%%%%%%%%%%%
% Создание макроса управления элементами, специфичными
% для вида работы (курс., бак., маг.)
% Здесь ничего не менять:
\usepackage{ifthen}
\newcounter{worktype}
\newcommand{\typeOfWork}[1]
{
	\setcounter{worktype}{#1}
}

% ВНИМАНИЕ!
% Укажите тип работы: 0 - курсовая, 1 - бак., 2 - маг.,
% 3 - бакалаврская с главами.
\typeOfWork{1}
% Считается, что курсовая и бак. бьются на разделы (section) и
% подразделы (subsection), а маг. — на главы (chapter), разделы и
%  подразделы. Если хочется,
% чтобы бак. была с главами (например, если она большая),
% надо выбрать опцию 3.

% Если при выборе 2 или 3 вы забудете поменять класс
% документа на scrreprt (см. выше, в самом начале),
% то получите ошибку:
% ./aux/appearance.tex:52: Package scrbase Error: unknown option ` chapterprefix=

%%%%%%%%%%%%%%%%%%%%%%%%%%%%%%%%%%%%%%%%%%%%%%%%%%%%%%%%%%%%%%%%%%%%%%%%
% Информация об авторе и работе для титульной страницы

\usepackage {titling}

% Имя автора в именительном падеже (для маг.)
\newcommand {\me}{%
И.\,И.~Иванов%
}

% Имя автора в родительном падеже (для курсовой и бак.)
\newcommand {\byme}{%
И.\,И.~Иванова%
}

% Любимый научный руководитель
\newcommand{\supervisor}%
{учёная степень, учёное звание /  должность И. О. Фамилия}

% идентифицируем пол (только для курсовой и бак.)
\newcommand{\bystudent}{
Студента %Студентки % Для курсовой: с большой буквы
}

% Год публикации
\date{2015}

% Название работы
\title{Потенциально длинное название работы\\на две или три строки}

% Кафедра
%
\newboolean{needchair}
\setboolean{needchair}{false} % на ФИИТ не пишется (false), на ПМИ есть (true)

\newcommand {\thechair} {%
Кафедра компьютерного и аналогового моделирования светлого будущего%
}

\newcommand {\direction} {%
Направление подготовки\\
Фундаментальная информатика и информационные технологии%
}% Прикладная математика и информатика

%%%%%%%%%%%%%%%%%%%%%%%%%%%%%%%%%%%%%%%%%%%%%%%%%%%%%%%%%%%%%%%%%%%%%%%%
% Другие настраиваемые элементы текста

% Листинги с исходным кодом программ: укажите язык программирования
\usepackage{listings}
\lstset{
    language=[ISO]C++,%  Язык указать здесь
    basicstyle=\small\ttfamily,
    breaklines=true,%
    showstringspaces=false%
    inputencoding=utf8x%
}
% полный список языков, поддерживаемых данным пакетом, есть,
% например, здесь (стр. 13):
% ftp://ftp.tex.ac.uk/tex-archive/macros/latex/contrib/listings/listings.pdf

% Нумерация списков: можно при необходимести
% изменять вид нумерации (например, добавлять правую скобку).
% По умолчанию буду списки вида:
% 1.
% 2.
% Изменять вид нумерации можно в начале нумерации:
% \begin{enumerate}[1)] (В квадратных скобках указан желаемый вид)
\usepackage[shortlabels]{enumitem}
                    \setlist[enumerate, 1]{1.}

% Гиперссылки: настройте внешний вид ссылок
\usepackage%
[pdftex,unicode,pdfborder={0 0 0},draft=false,%backref=page,
    hidelinks, % убрать, если хочется видеть ссылки: это
               % удобно в PDF файле, но не должно появиться на печати
    bookmarks=true,bookmarksnumbered=false,bookmarksopen=false]%
{hyperref}


\usepackage {amsmath}      % Больше математики
\usepackage {amssymb}
\usepackage {textcase}     % Преобразование к верхнему регистру
\usepackage {indentfirst}  % Красная строка первого абзаца в разделе

\usepackage {fancyvrb}     % Листинги: определяем своё окружение Verb
\DefineVerbatimEnvironment% с уменьшенным шрифтом
	{Verb}{Verbatim}
	{fontsize=\small}

% Вставка рисунков
\usepackage {graphicx}

% Общее оформление
% ----------------------------------------------------------------
% Настройка внешнего вида

%%% Шрифты

% если закомментировать всё — консервативная гарнитура Computer Modern
\usepackage{paratype} % профессиональные свободные шрифты
%\usepackage {droid}  % неплохие свободные шрифты от Google
%\usepackage{mathptmx}
%\usepackage {mmasym}
%\usepackage {psfonts}
%\usepackage{lmodern}
%var1: lh additions for bold concrete fonts
%\usepackage{lh-t2axccr}
%var2: the package below could be covered with fd-files
%\usepackage{lh-t2accr}
%\usepackage {pscyr}

% Геометрия текста

\usepackage{setspace}       % Межстрочный интервал
\onehalfspacing

\newlength\MyIndent
\setlength\MyIndent{1.25cm}
\setlength{\parindent}{\MyIndent} % Абзацный отступ
\frenchspacing            % Отключение лишних отступов после точек
\KOMAoptions{%
    DIV=calc,         % Пересчёт геометрии
    numbers=endperiod % точки после номеров разделов
}

                            % Консервативный вариант:
%\usepackage                % ручное задание геометрии
%[%                         % (не рекомендуется в проф. типографии)
%  margin = 2.5cm,
  %includefoot,
  %footskip = 1cm
%] %
%  {geometry}

%%% Заголовки

\ifthenelse{\equal{\theworktype}{2}}{%
\KOMAoptions{%
    numbers=endperiod,% точки после номеров разделов
    headings=normal,   % размеры заголовков поменьше стандартных
    chapterprefix=true,% Печатать слово Глава в магистерской
    appendixprefix=true% Печатать слово Приложение
}
}

% шрифт для оформления глав и названия содержания
\newcommand{\SuperFont}{\Large\sffamily\bfseries}

% Заголовок главы
\ifthenelse{\value{worktype} > 1}{%
\renewcommand{\SuperFont}{\Large\normalfont\sffamily}
\newcommand{\CentSuperFont}{\centering\SuperFont}
\usepackage{fncychap}
\ChNameVar{\SuperFont}
\ChNumVar{\CentSuperFont}
\ChTitleVar{\CentSuperFont}
\ChNameUpperCase
\ChTitleUpperCase
}

% Заголовок (под)раздела с абзацного отступа
\addtokomafont{sectioning}{\hspace{\MyIndent}}

\renewcommand*{\captionformat}{~---~}
\renewcommand*{\figureformat}{Рисунок~\thefigure}

% Плавающие листинги
\usepackage{float}
\floatstyle{ruled}
\floatname{ListingEnv}{Листинг}
\newfloat{ListingEnv}{htbp}{lol}[section]

% точка после номера листинга
\makeatletter
\renewcommand\floatc@ruled[2]{{\@fs@cfont #1.} #2\par}
\makeatother


%%% Оглавление
\usepackage{tocloft}

% шрифт и положение заголовка
\ifthenelse{\value{worktype} > 1}{%
\renewcommand{\cfttoctitlefont}{\hfil\SuperFont\MakeUppercase}
}{
\renewcommand{\cfttoctitlefont}{\hfil\SuperFont}
}

% слово Глава
\usepackage{calc}
\ifthenelse{\value{worktype} > 1}{%
\renewcommand{\cftchappresnum}{Глава }
\addtolength{\cftchapnumwidth}{\widthof{Глава }}
}

% Очищаем оформление названий старших элементов в оглавлении
\ifthenelse{\value{worktype} > 1}{%
\renewcommand{\cftchapfont}{}
\renewcommand{\cftchappagefont}{}
}{
\renewcommand{\cftsecfont}{}
\renewcommand{\cftsecpagefont}{}
}

% Точки после верхних элементов оглавления
\renewcommand{\cftsecdotsep}{\cftdotsep}
%\newcommand{\cftchapdotsep}{\cftdotsep}

\ifthenelse{\value{worktype} > 1}{%
    \renewcommand{\cftchapaftersnum}{.}
}{}
\renewcommand{\cftsecaftersnum}{.}
\renewcommand{\cftsubsecaftersnum}{.}
\renewcommand{\cftsubsubsecaftersnum}{.}

%%% Списки (enumitem)

\usepackage {enumitem}      % Списки с настройкой отступов
\setlist %
{ %
  leftmargin = \parindent, itemsep=.5ex, topsep=.4ex
} %

% По ГОСТу нумерация должны быть буквами: а, б...
%\makeatletter
%    \AddEnumerateCounter{\asbuk}{\@asbuk}{м)}
%\makeatother
%\renewcommand{\labelenumi}{\asbuk{enumi})}
%\renewcommand{\labelenumii}{\arabic{enumii})}

%%% Таблицы: выбрать более подходящие

\usepackage{booktabs} % считаются наиболее профессионально выполненными
%\usepackage{ltablex}
%\newcolumntype {L} {>{---}l}

%%% Библиография

\usepackage{csquotes}        % Оформление списка литературы
\usepackage[
  backend=biber,
  hyperref=auto,
  sorting=none, % сортировка в порядке встречаемости ссылок
  language=auto,
  citestyle=gost-numeric,
  bibstyle=gost-numeric
]{biblatex}
\addbibresource{biblio.bib} % Файл с лит.источниками

% Настройка величины отступа в списке
\ifthenelse{\value{worktype} < 2}{%
\defbibenvironment{bibliography}
  {\list
     {\printtext[labelnumberwidth]{%
    \printfield{prefixnumber}%
    \printfield{labelnumber}}}
     {\setlength{\labelwidth}{\labelnumberwidth}%
      \setlength{\leftmargin}{\labelwidth}%
      \setlength{\labelsep}{\dimexpr\MyIndent-\labelwidth\relax}% <----- default is \biblabelsep
      \addtolength{\leftmargin}{\labelsep}%
      \setlength{\itemsep}{\bibitemsep}%
      \setlength{\parsep}{\bibparsep}}%
      \renewcommand*{\makelabel}[1]{\hss##1}}
  {\endlist}
  {\item}
}{}

% ----------------------------------------------------------------
% Настройка переносов и разрывов страниц

\binoppenalty = 10000      % Запрет переносов строк в формулах
\relpenalty = 10000        %

\sloppy                    % Не выходить за границы бокса
%\tolerance = 400          % или более точно
\clubpenalty = 10000       % Запрет разрывов страниц после первой
\widowpenalty = 10000      % и перед предпоследней строкой абзаца

% ----------------------------


% Стили для окружений типа Определение, Теорема...
% Оформление теорем (ntheorem)

\usepackage [thmmarks, amsmath] {ntheorem}
\theorempreskipamount 0.6cm

\theoremstyle {plain} %
\theoremheaderfont {\normalfont \bfseries} %
\theorembodyfont {\slshape} %
\theoremsymbol {\ensuremath {_\Box}} %
\theoremseparator {:} %
\newtheorem {mystatement} {Утверждение} [section] %
\newtheorem {mylemma} {Лемма} [section] %
\newtheorem {mycorollary} {Следствие} [section] %

\theoremstyle {nonumberplain} %
\theoremseparator {.} %
\theoremsymbol {\ensuremath {_\diamondsuit}} %
\newtheorem {mydefinition} {Определение} %

\theoremstyle {plain} %
\theoremheaderfont {\normalfont \bfseries} 
\theorembodyfont {\normalfont} 
%\theoremsymbol {\ensuremath {_\Box}} %
\theoremseparator {.} %
\newtheorem {mytask} {Задача} [section]%
\renewcommand{\themytask}{\arabic{mytask}}

\theoremheaderfont {\scshape} %
\theorembodyfont {\upshape} %
\theoremstyle {nonumberplain} %
\theoremseparator {} %
\theoremsymbol {\rule {1ex} {1ex}} %
\newtheorem {myproof} {Доказательство} %

\theorembodyfont {\upshape} %
%\theoremindent 0.5cm
\theoremstyle {nonumberbreak} \theoremseparator {\\} %
\theoremsymbol {\ensuremath {\ast}} %
\newtheorem {myexample} {Пример} %
\newtheorem {myexamples} {Примеры} %

\theoremheaderfont {\itshape} %
\theorembodyfont {\upshape} %
\theoremstyle {nonumberplain} %
\theoremseparator {:} %
\theoremsymbol {\ensuremath {_\triangle}} %
\newtheorem {myremark} {Замечание} %
\theoremstyle {nonumberbreak} %
\newtheorem {myremarks} {Замечания} %


% Титульный лист
% Макросы настройки титульной страницы
% В этот файл не предполагается вносить изменения

%\usepackage {showframe}

% Вертикальные отступы на титульной странице
\newcommand{\vgap}{\vspace{16pt}}

% Помещение города и даты в нижний колонтитул
\usepackage{scrlayer}
\DeclareNewLayer[
  foot,
  foreground,
  contents={%
    \raisebox{\dp\strutbox}[\layerheight][0pt]{%
      \parbox[b]{\layerwidth}{\centering Ростов-на-Дону\\ \thedate%
       \\\mbox{}
       }}%
  }
]{titlepage.foot.fg}
\DeclareNewPageStyleByLayers{titlepage}{titlepage.foot.fg}


\AtBeginDocument %
{ %
  %
  \begin{titlepage}
  %
    \thispagestyle{titlepage}

    {\centering
    %
    \MakeTextUppercase {МИНИСТЕРСТВО ОБРАЗОВАНИЯ И НАУКИ РФ}

    \vgap

    Федеральное государственное автономное образовательное\\
    учреждение высшего образования\\
    \MakeTextUppercase {Южный федеральный университет}

    \vgap

	Институт математики, механики и компьютерных наук
    имени~И.\,И.\,Воровича

    \vgap

    \direction

    \ifthenelse{\boolean{needchair}}{
    \vgap

    \thechair}{}

    \vspace* {\fill}

    \ifthenelse{\value{worktype} = 2}{%
    \me

    \vgap}{}

    {\usefont{T2A}{PTSansCaption-TLF}{m}{n}
    \MakeTextUppercase{\thetitle}}

    \ifthenelse{\value{worktype} = 2}{%
     \vgap

    Магистерская диссертация}{}
    \ifthenelse{\value{worktype} = 0}{
     \vgap

    Курсовая работа
    }{}%
    \ifthenelse{\value{worktype} = 1 \OR \value{worktype} = 3}{
     \vgap

    Выпускная квалификационная работа\\
    на степень бакалавра
    }{}%

    \vspace {\fill}

    \begin{flushright}
    \ifthenelse{\value{worktype} = 0 \OR 
                \value{worktype} = 1 \OR
                \value{worktype} = 3}{
      \bystudent \ifthenelse{\value{worktype} = 0}{3}{4}\ курса\\
      \byme
    }{}

    \vgap

    Научный руководитель:\\
    \supervisor\\
    \ifthenelse{\value{worktype} = 2}{%
    Рецензент:\\
    ученая степень, ученое звание, должность
    И. О. Фамилия
    }{}
	\end{flushright}
    \ifthenelse{\value{worktype} = 0}{
    \vspace{\fill}
            \begin{flushleft}
              \begin{tabular}{cc}
                \underline{\hspace{4cm}}&\underline{\hspace{5cm}}\\
                {\small оценка (рейтинг)} & {\small  подпись руководителя}\\
              \end{tabular}
            \end{flushleft}
    }{}
  	\vspace {\fill}
  %Ростов-на-Дону

    %\thedate

  }\end{titlepage}
  %
  %
  \tableofcontents
  %
  \clearpage
} %



% Команды для использования в тексте работы


% макросы для начала введения и заключения
\newcommand{\Intro}{\addsec{Введение}}
\ifthenelse{\value{worktype} > 1}{%
    \renewcommand{\Intro}{\addchap{Введение}}%
}

\newcommand{\Conc}{\addsec{Заключение}}
\ifthenelse{\value{worktype} > 1}{%
    \renewcommand{\Conc}{\addchap{Заключение}}%
}

% Правильные значки для нестрогих неравенств и пустого множества
\renewcommand {\le} {\leqslant}
\renewcommand {\ge} {\geqslant}
\renewcommand {\emptyset} {\varnothing}

% N ажурное: натуральные числа
\newcommand {\N} {\ensuremath{\mathbb N}}

% значок С++ — используйте команду \cpp
\newcommand{\cpp}{%
C\nolinebreak\hspace{-.05em}%
\raisebox{.2ex}{+}\nolinebreak\hspace{-.10em}%
\raisebox{.2ex}{+}%
}

% Неразрывный дефис, который допускает перенос внутри слов,
% типа жёлто-синий: нужно писать жёлто"/синий.
\makeatletter
    \defineshorthand[russian]{"/}{\mbox{-}\bbl@allowhyphens}
\makeatother


\endinput

% Конец файла

\graphicspath{ {img/} }

\NewBibliographyString{langjapanese}
\NewBibliographyString{fromjapanese}

\begin{document}

\Intro
В современном мире постоянно усиливаются потоки информации, представляющей собой различного рода структурированные данные. Актуальным представляется вопрос обработки этих потоков. Для решения этой задачи появляются всё новые форматы представления данных. Одним из наиболее популярных форматов обмена структурированными данными является JSON (JavaScript Object Notation).

Постоянное изменение форматов данных вслед за меняющимся миром должно отражаться в системах типов языков программирования, которые используются для создания программ обработки потоков информации. В языках программирования со статической типизацией сложней отражать указанную изменчивость: малейшее изменение формата водных данных может повлечь необходимость больших изменений в коде для того, чтобы он стал, во-первых, компилируемым, во-вторых, рабочим.

Для статически типизированных языков программирования хотелось бы добавить гибкость в указанном отношении, это значит, что следует добавить возможность автоматизированного анализа источников данных и генерации типов на основе этого анализа. 

В 2011 году была представлена концепция «поставщиков типов» (\lstinline{Type Providers}) в языке \lstinline{F#}. Поставщики типов являются компонентами, которые включают в программу новые методы и типы, основанные на входных данных из внешних источников данных. То есть позволяют программисту работать напрямую с данными без определения дополнительных функций для получения и обработки данных. При этом, удобство придает и то, что поставщики типов являются плагинами к компилятору и плагинами к \lstinline{IDE}. Примеры поставщиков: данные в виде JSON, HTML, регулярных выражений и прочие.

Поставщики типов показали свою эффективность в ряде прикладных задач ~\cite{typo}. Возникает вопрос о переносе принципа работы поставщиков типов на другие функциональные языки.

Цель данной работы --- создание механизма для анализа входных данных JSON и генерации определений типов функционального языка программирования Haskell на основе проведённого анализа.

Для достижения данной цели в работе поставлены следующие задачи.
\begin{itemize}
  \item Получение абстрактного синтаксического дерева (далее --- AST) по исходному \lstinline{JSON}.
  \item Преобразование AST в структуру, пригодную для генерации алгебраического типа данных \lstinline{Haskell}.
  \item Использование сгенерированного типа данных в коде.
\end{itemize}

\chapter{Предварительные сведения}

\section{JavaScript Object Notation}

JSON (\lstinline{JavaScript Object Notation})~\cite{jsonStandart} --- простой формат обмена данными, основанный на подмножестве языка программирования \lstinline{JavaScript}. При этом рассматривемый формат независим от реализации и может использоваться любым языком программирования. Файл в формате \lstinline{JSON} представляет собой неупорядоченное множество пар ключ-значение, значения которого могут иметь следующий тип:  

\begin{itemize}
  \item объекты (выделяются \{ ... \}),
  \item массивы (выделяются [ ... ]),
  \item строки,
  \item логические выражения (true | false),
  \item null-значения.
\end{itemize}

\begin{ListingEnv}[H]
\begin{Verb}
{     
    "firstName": "John",
    "lastName" : "Smith",
    "age" : 25
}
\end{Verb}
\caption{Пример данных в формате JSON}
\label{listing:jsonExample}
\end{ListingEnv}

Наглядно продемонстрировать вышеобозначенную структуру можно при помощи схем, представленных на рисунках ~\ref{fig:objectGr}, ~\ref{fig:arrayGr} и ~\ref{fig:valueGr}.

\begin{figure}[!ht]
\centering
\includegraphics[width=\textwidth]{object}
\caption{\label{fig:objectGr}Объект}
\end{figure}

\begin{figure}[!ht]
\centering
\includegraphics[width=\textwidth]{array}
\caption{\label{fig:arrayGr}Массив}
\end{figure}

\begin{figure}[!ht]
\centering
\includegraphics[width=\textwidth]{value}
\caption{\label{fig:valueGr}Значение}
\end{figure}

\section{Алгебраические типы данных}

Алгебраические типы данных (далее --- АТД) --- вид составных типов, представленных типом-произведением, типом-суммой, либо комбинацией: суммой произведений.~\cite{haskellGreatGood} Последний вариант можно проиллюстрировать на примере двоичного дерева:

\begin{lstlisting}[language=Haskell]
data Tree a = Leaf a
          | Node (Tree a) (Tree a)
\end{lstlisting}
В примере \lstinline{Tree} является суммой произведений. Сумма определяется знаком «|». А произведение типов простым перечислением (\lstinline{Node (Tree a) (Tree a)}) 

Также существует другой вариант определения типа, который называется синтаксисом записи с именованными полями:

\begin{lstlisting}[language=Haskell]
data Person = Person { firstName :: String
                       , lastName :: String
                       , age :: Int }
\end{lstlisting}

Вместо простого перечисления типов созданная структура наполняется полями и их значениями. Главное преимущество заключается в том, что подобный синтаксис генерирует функции для извлечения полей. Также облегчается чтение и понимание типа данных. 

Стоит отметить, что с помощью АТД и стандартных типов \lstinline{Haskell} (например, строковых String, списковых [a] и т. д.) можно представить \lstinline{JSON}-структуру. 

АТД являются фундаментальным понятием. Простота и гибкость таких типов открывают дорогу таких техникам программирования как обобщенное программирование, метапрограммирование и т. д. 

\section{Монада State}

Монада \lstinline{State} применима в тех случаях, когда имеется некоторое состояние, которое подвергается постоянным изменениям. Cтоит отметить, что при таком способе мы трансформируем состояние, но при этом не теряем «чистоту» \: функций. 

\subsection{Control.Monad.State}

Стандартный модуль Control.Monad.State~\cite{stateControl} определяет тип, оборачивающий вычисление с состоянием:

\begin{lstlisting}[language=Haskell]
newtype State s a = State {runState :: s -> (a,s) }
\end{lstlisting} 
Как следует из определения, вычисление в монаде State возвращает некоторый результат и при этом в случае необходимости меняет состояние. Операции с состояниями реализованы следующими функциями: \lstinline{get}\:(получает состояние) и \lstinline{put}\:(изменяет состояние на заданное). В листинге~\ref{listing:stateGetPut} приведен простой пример, позволяющий продемонстрировать возможности монады \lstinline{State}.

\begin{ListingEnv}[H]
\begin{Verb}
tick :: State Int Int
tick = do n <- get
	  put (n+1)
          return n

ghci> runState tick 3
(3,4)
\end{Verb}
\caption{Пример использования монады State}
\label{listing:stateGetPut}
\end{ListingEnv}
Вычисление в монаде запускается с помощью \lstinline{runState} и передаваемого состояния. В функции \lstinline{tick} происходит получение входного значения \lstinline{3} (\lstinline{get}), далее меняется состояние (\lstinline{put}) и на выходе возвращается значение(\lstinline{return}).

Также модуль содержит и другие полезные функции для работы с состояниями. Некоторые из них представлены в листинге~\ref{listing:stateFunc}. Функции с суффиксом -State отличаются типом возвращаемого значения. Необходимое поведение можно выбрать исходя из названия функции.

\begin{ListingEnv}[H]
\begin{Verb}
modify :: MonadState s m => (s -> s) -> m ()

execState:: State s a -> s -> s

runState:: State s a -> s -> (a, s)

evalState:: State s a -> s -> a
\end{Verb}
\caption{Функции модуля Control.Monad.State}
\label{listing:stateFunc}
\end{ListingEnv}

Например, функция \lstinline{modify} преобразует внутреннее состояние функцией, которую получает на вход. Можно реализовать код листинга ~\ref{listing:stateGetPut} через \lstinline{modify} (листинг~\ref{listing:modifyState}).

\begin{ListingEnv}[H]
\begin{Verb}
tick :: State Int Int
tick = do modify (+1)
          return n

ghci> runState tick 3
(3,4)
\end{Verb}
\caption{Функция modify модуля Control.Monad.State}
\label{listing:modifyState}
\end{ListingEnv}
	
\section{Библиотека Data.Aeson}
\label{sec:secAeson}

\subsection{Обзор}

\lstinline{Data.Aeson} --- библиотека для работы с файлами в формате \lstinline{JSON}, написанная на языке \lstinline{Haskell}.~\cite{aeson} В данной  библиотеке используются два основных класса типов --- \lstinline{FromJSON} и \lstinline{ToJSON}.~\cite{aesonEx} Типы, имеющие возможность кодирования/декодирования, должны быть экземплярами классов \lstinline{FromJSON}, \lstinline{ToJSON}. Самый простой способ использования библиотеки заключается в определении типов данных и экземпляров \lstinline{FromJSON}, \lstinline{ToJSON}. 

Существует возможность определить экземпляры для кодирования/декодирования по умолчанию благодаря инструкции компилятора \lstinline{DeriveGeneric} и экземпляру \lstinline{Generic} для кодируемых/декодируемых типов. Рассмотрим листинг~\ref{listing:genericData}, демонстрирующий данную возможность с условным типом данных. 

\begin{ListingEnv}[H]
\begin{Verb}
{-# LANGUAGE DeriveGeneric #-}
import GHC.Generics

data DataName = DataName {
        field1::Int,
        field2::Int
     } deriving (Generic, Show)

instance ToJSON DataName

instance FromJSON DataName
\end{Verb}
\caption{Создание экземпляров по умолчанию}
\label{listing:genericData}
\end{ListingEnv}

\subsection{Работа с AST}

\lstinline{Data.Aeson} имеет свой собственный тип для представления конвертируемого \lstinline{JSON}-файла. Этот тип называется \lstinline{Value} и имеет шесть конструкторов значения:

\begin{ListingEnv}[H]
\begin{Verb}
data Value
  = Object Object
  | Array Array
  | String Text
  | Number Scientific
  | Bool Bool
  | Null
\end{Verb}
\caption{Конструкторы Value}
\label{listing:value}
\end{ListingEnv}

Таким образом \lstinline{Aeson} позволяет получить абстрактное синтаксическое дерево по \lstinline{JSON}. Это бывает полезно в случаях, когда неизвестен тип данных, соответствующий входному файлу. Имея AST, можно написать функцию для его обхода.~\cite{aesonEx}

В качестве примера рассмотрим получение AST для двух случаев: простой (листинг~\ref{listing:astGetSimple}) и более сложный со вложенными объектами (листинг~\ref{listing:astGetComp}).

\begin{ListingEnv}[H]
\begin{Verb}
decode :: FromJSON a => ByteString -> Maybe a

ghci> decode "{\"foo\": 123}" :: Maybe Value
Just (Object (fromList [("foo",Number 123)]))
\end{Verb}
\caption{JSON без вложенных объектов}
\label{listing:astGetSimple}
\end{ListingEnv}

\begin{ListingEnv}[H]
\begin{Verb}
ghci> decode "{\"foo\": [\"abc\",\"def\"]}" :: Maybe Value
Just (Object (fromList [("foo", Array (fromList [String "abc", 
                                               String "def"]))]))
\end{Verb}
\caption{JSON со вложенными объектами}
\label{listing:astGetComp}
\end{ListingEnv}

\section{Template Haskell}

\lstinline{Template Haskell} --- это расширение языка \lstinline{Haskell}, реализующее средства для метапрограммирования.~\cite{extensionHub} Оно позволяет использовать \lstinline{Haskell} одновременно как язык исходный, так и целевой. Поскольку работа ведется с расширением, в исполняемый файл необходимо добавить директиву:

\begin{lstlisting}[language=Haskell]
{-# LANGUAGE TemplateHaskell #-}.
\end{lstlisting}

\subsection{Монада Q}

Монада \lstinline{Q} оборачивает значения типов, предназначенных для последующей генерации других конструкций на языке \lstinline{Haskell}.~\cite{tempHaskell} Такие типы полностью удовлетворяют синтаксису языка и представляют собой абстрактное синтаксическое дерево кода на \lstinline{Haskell}: тип \lstinline{Exp} --- для генерации выражений, \lstinline{Dec} --- для объявлений, \lstinline{Lit} --- для литералов и т.д.~\cite{thSyntax}

\subsection{Модуль Language.Haskell.TH.Syntax}

Тип \lstinline{Exp} определен в модуле \lstinline{Language.Haskell.TH.Syntax}~\cite{coverHaskell} и  \lstinline{Exp} имеет семнадцать конструкторов значения. В листинге ~\ref{listing:expConstr} представлены первые четыре из них.  

\begin{ListingEnv}[H]
\begin{Verb}
data Exp
       = VarE Name
       | AppE Exp Exp
       | MultiIfE [(Guard, Exp)]
       | CondE Exp Exp Exp
       | ...
\end{Verb}
\caption{Конструкторы Exp}
\label{listing:expConstr}
\end{ListingEnv}

\subsection{Вклейка (splicing)}

Вклейка кода --- преобразование шаблона (структура \lstinline{Template Haskell}) с данным параметром в обычный \lstinline{Haskell}-код во время компиляции и вклеивает его на то же место. Вклейка производится оператором  \lstinline{\$(...)}. Важно, чтобы между скобками и оператором \lstinline{\$} не было пробелов. Также существует ограничение по использованию вклейки: ее можно применить только из другого модуля, т.е. отсутствует возможность "вклейки" шаблона в одном модуле вместе с его определением. Вклейка используется в разных компонентах языка (см. таблицу~\ref{tab:splicing}).

\begin{table}
\centering
\caption{\label{tab:splicing}Использование вклейки}
\begin{tabular}{llr}
\toprule
\cmidrule(r){1-2}
Где используется  & Тип    \\
\midrule
Выражения          & Q Exp           \\
Объявления верхнего уровня      & Q [Dec]          \\
Типы       & Q Type          \\
\bottomrule
\end{tabular}
\end{table}

\chapter{Генерация АТД по JSON-файлу}

\section{Выбор средств для реализации}

Изначально при реализации задачи, поставленной во введении, стоял выбор между несколькими средствами. Рассмотрим каждое из них.

\subsection{Расширение к языку Haskell}
При подобной реализации становятся очевидными некоторые преимущества, связанные с простотой использования (необходимо лишь прописать директиву, добавляющую это расширение), но при этом она представляется сложной в реализации, поскольку для ее осуществления необходимо изменить исходные коды компилятора  \lstinline{GHC}. При этом в данной реализации не хватало бы наглядности и синтаксис отличался бы от привычного.

\subsection{Написание парсера}
Данное решение позволяет использовать стандартные методы и наработки, связанные с парсерами, что предоставляет возможность лучше контролировать представление для генерации. При этом можно вывести готовый алгебраический тип данных. Но в данном случае возникает проблема: как использовать этот тип незамедлительно без дополнительных манипуляций.   	

\subsection{Template Haskell}
Данный подход оказался предпочтительнее остальных по ряду причин. Во-первых, при его использовании отсутствуют недостатки, присущие другим подходам, в том числе существует возможность использования сгенерированного типа без дополнительных накладных ресурсов и средств. Во-вторых, предоставляется возможность генерирования данных  и последующего незамедлительного их использования. Для получение AST будет использоваться ранее упомянутая библиотека \lstinline{Data.Aeson}. Помимо этого, данная идея представляется более гибкой с точки зрения практической реализации. Возможно, аналогичные средства дают некоторые преимущества, но они не так существенны с точки зрения эффективности.

\section{Программная реализация генератора АТД}

\subsection{Применение DataD (Template Haskell)}

Программная реализация средствами \lstinline{Template Haskell} генерирует объявление типов данных: используется конструктор значения \lstinline{DataD} типа \lstinline{Dec}\:(листинг ~\ref{listing:dataDTH}). \lstinline{DataD} является конструктором значения \lstinline{Dec}. Тип \lstinline{Cxt} определяет классы типов, которым должны принадлежать различные типы, входящие в определение АТД.

\lstinline{Name} --- абстрактный тип, представляющий имена в синтаксическом дереве. Используется для определения имени полей и конструкторов. Функция \lstinline{mkName}\:(листинг~\ref{listing:mkName}) создает значение типа \lstinline{Name} из обычной строки (\lstinline{String}), с ее содержанием в качестве имени.

\begin{ListingEnv}[H]
\begin{Verb}
Dec 
   = DataD Cxt Name [TyVarBndr] [Con] Cxt
   | ...
\end{Verb}
\caption{Конструктор значения типа Dec}
\label{listing:dataDTH}
\end{ListingEnv} 

\begin{ListingEnv}[H]
\begin{Verb}
mkName :: String -> Name
\end{Verb}
\caption{Особенность: чистота функции mkName}
\label{listing:mkName}
\end{ListingEnv} 

В коде реализации используется тип из листинга~\ref{listing:dataDTH} для вклейки в объявлениях верхнего уровня. Тип \lstinline{RecC} дает понять, что сгенерированный тип должен представлять собой запись с именованными полями. Использование класса типов \lstinline{Generic} понадобится в будущем, когда будут генерироваться экземпляры классов \lstinline{FromJSON} и \lstinline{ToJSON} для вклеиваемого типа (см. раздел~\ref{sec:secAeson}).

\begin{ListingEnv}[H]
\begin{Verb}
DataD
     []
     (mkName $ firstLetterToUpper key')
     []
     [ RecC (mkName $ firstLetterToUpper key')  (result) ]
     [mkName "Generic", mkName "Show", mkName "Eq"]
\end{Verb}
\caption{Генерация Data в тексте программы}
\label{listing:dataDTHjson}
\end{ListingEnv} 

Важно помнить, что имена типов данных в \lstinline{Haskell} начинаются с заглавных букв. Для соблюдения этого правила определим функцию \lstinline{firstLetterToUpper} (листинг~\ref{listing:letterChange}).

\begin{ListingEnv}[H]
\begin{Verb}
firstLetterToUpper :: String -> String
firstLetterToUpper (x:xs) = (Data.Char.toUpper $ x) : (xs)
\end{Verb}
\caption{Функция смены первой буквы на заглавную}
\label{listing:letterChange}
\end{ListingEnv} 

\section{Использование монады State}

Для аккумуляции типов \lstinline{Dec} был необходим аналог глобальной переменной в императивных языках программирования. В функциональных языках, как правило, отсутствуют подобные средства в чистом виде, однако похожего поведения можно добиться, используя монаду \lstinline{State}:

\begin{lstlisting}[language=Haskell]
State [Dec] ()
\end{lstlisting}

Накапливаются типы для генерации с помощью функции \lstinline{modify} из модуля \lstinline{Control.Monad.State}.~\cite{stateM} Однако существует проблема: каждый раз, когда используется \lstinline{modify}, порождается новый экземпляр типа \lstinline{State [Dec] ()}, который теряет накопленные результаты обхода, то есть условно при таком подходе не будет глобальной переменной, в которую собираются все необходимые данные. 

Для того чтобы решить возникшую проблему, нам необходима монадическая свертка по ключу и значению (листинг~\ref{listing:foldMonad}). Свертка реализована с помощью \lstinline{foldlM} из модуля \lstinline{Data.Foldable} и функции \lstinline{uncurry} (преобразует каррированную функцию в функцию, принимающую пару).

Еще одно преимущество использования функционального языка \lstinline{Haskell} заключается в том, что существуют полезные средства для поиска необходимых функций --- поиск по сигнатуре. Данную возможность реализует сервис \lstinline{Hayoo}~\cite{hayoo}. C помощью него была найдена \lstinline{foldlWithKeyM}. 

\begin{ListingEnv}[H]
\begin{Verb}
import qualified Data.HashMap.Strict as StrHash
import qualified Data.Foldable    as FB

foldlWithKeyM :: (Monad m) => (b -> k -> a -> m b) -> b ->
                                       StrHash.HashMap k a -> m b
foldlWithKeyM f b = FB.foldlM f' b . StrHash.toList
  where f' a = uncurry (f a)
\end{Verb}
\caption{foldlWithKeyM}
\label{listing:foldMonad}
\end{ListingEnv} 

\section{Формирование АТД}

Основной функцией, которая запускает проход по AST и накапливает значения в монаде \lstinline{State} является: 

\begin{lstlisting}[language=Haskell]
convertObject:: String -> Value -> State [Dec] ()
\end{lstlisting}

Функция \lstinline{convertObject} получает на вход имя типа данных и AST, полученное с помощью \lstinline{decode}\:(\lstinline{Data.Aeson}). Затем происходит спуск по дереву и анализ его вершин:

\begin{lstlisting}[language=Haskell]
convertFields:: MonadState [Dec] m => Value -> 
                                         m [(Name, Strict, Type)]
\end{lstlisting}

Как следует из сигнатуры, анализ и спуск происходит в монаде. Именно при проходе по вершинам мы используем написанный ранее \lstinline{foldlWithKeyM}. Главная задача --- учитывать, что при обходе мы можем встретить вложенные в объект объекты и вызвать функцию рекурсивно. Для этого мы используем функцию \lstinline{isObject} (листинг~\ref{listing:objectVal}). \lstinline{convertFields} получает массив из \lstinline{(Name, Strict, Type)}, который подставляется в (листинг~\ref{listing:dataDTHjson}) вместо переменной \lstinline{result}.

\begin{ListingEnv}[H]
\begin{Verb}
isObject :: Value -> Bool
isObject (Object obj) = True;
isObject _ = False;
\end{Verb}
\caption{Проверка на принадлежность Object}
\label{listing:objectVal}
\end{ListingEnv} 

Все необходимые для генерации данные собираются за один проход AST. При анализе дерева, в том случае если алгоритм доходит до листового узла, вызывается \lstinline{return}, который возвращает \lstinline{[(Name, Strict, Type)]} (листинг~\ref{listing:otherwiseTree}). И далее продолжается обход. \lstinline{MState} --- квалифицированый импорт модуля \lstinline{Control.Monad.State}.

\begin{ListingEnv}[H]
\begin{Verb}
do
      (MState.return (((mkName $  key'), NotStrict,
                                (mkValType val' key') ) : list'))
\end{Verb}
\caption{Простой случай при обходе}
\label{listing:otherwiseTree}
\end{ListingEnv} 

Если при проходе по дереву срабатывает условие из листинга~\ref{listing:otherwiseTree}, происходит рекурсивный спуск --- вызов функции \lstinline{convertFields} с вложенным найденным объектом (листинг ~\ref{listing:isObjectTree}).

\begin{ListingEnv}[H]
\begin{Verb}
do
      result <- convertFields $ val'
      MState.modify ((Prelude.++) [DataD ... ])
      (MState.return (((mkName $  key'), NotStrict,
                                (mkValType val' key') ) : list'))
\end{Verb}
\caption{Сложный случай при обходе}
\label{listing:isObjectTree}
\end{ListingEnv} 

Таким образом была описана вся логика программы. Обрабатывается каждый вложенный объект и каждый лист AST. На выходе получаем структуру \lstinline{[Dec]} и оборачиваем ее в монаду \lstinline{Q} для вызова вклейки из другого модуля через оператор \lstinline{\$(...)}.

\chapter{Пример использования полученного инструмента}

\lstinline{Git}-репозиторий с исходным кодом на языке \lstinline{Haskell} доступен по адресу ~\cite{diploma}. Тестовые примеры запускались на компиляторе \lstinline{GHC} версии \lstinline{7.10.3}. Библиотека собрана в \lstinline{cabal}-пакет и также доступна ~\cite{diploma}. Для компиляции библиотеки нужно перейти в его каталог и выполнить стандартную команду:

\begin{lstlisting}[language=Haskell]
$ cabal install
\end{lstlisting}

\section{Отладка программ с Template Haskell}

Компилятор \lstinline{GHC} предоставляет мощные инструменты для отладки программы. ~\cite{debugGHC} К примеру, можно ставить контрольные точки, получать подробное описание ошибок, генерировать полезные структуры, смотреть на AST и т.д. Для этого используются флаги компилятору. 

Полезным флагом для программы, использующей \lstinline{Template Haskell} служит \lstinline{-ddump-splices}. Его средствами организован вывод полученного из шаблона выражения либо ошибки (информативное сообщение).

\begin{ListingEnv}[H]
\begin{Verb}
on.hs:8:3-17: Splicing declarations
    getDataFromJSON
  ======>
    data JSONData
      = JSONData {name :: String}
      deriving (Generic, Show, Eq)
\end{Verb}
\caption{Запуск отладчика с флагом -ddump-splices}
\end{ListingEnv}

\section{Простой пример}
Для начала будет рассмотрен простой пример. На вход программе подается простой (без вложенных объектов) \lstinline{JSON} (листинг ~\ref{listing:json1}).

\begin{ListingEnv}[H]
\begin{Verb}
{
    "name" : "Joe",
    "age" : 25,
    "avg" : 4,
    "arr" : [1,2,3]
}
\end{Verb}
\caption{Вход программы}
\label{listing:json1}
\end{ListingEnv}

В итоге мы получаем тип данных с именованными полями, полностью соответствующий постановленной задаче:

\begin{lstlisting}[language=Haskell]
data JSONData 
        = JSONData {arr :: [Float]
                    name :: String,
                    age :: Float,
                    avg :: Float}
          deriving (Show, Eq, Generic)
\end{lstlisting}

\section{Более сложный пример}

На вход программе подается \lstinline{JSON} со вложенным объектом (листинг ~\ref{listing:json2}).  Стоит отметить то, как он будет представлен. 

\begin{ListingEnv}[H]
\begin{Verb}
{
    "name" : "Joe",
    "age" : 25,
    "avg" : 4,
    "arra" : 
             {
                 "fg" : "JSONTest"  
             }
}
\end{Verb}
\caption{Вход программы}
\label{listing:json2}
\end{ListingEnv}

После выполнения программы мы получаем два типа данных, что полностью соответствует заявленным требованиям к программе. Первый тип в своем определении использует второй. 

\begin{lstlisting}[language=Haskell]
data JSONData
      = JSONData {name :: String,
                  arra :: Arra,
                  age :: Float,
                  avg :: Float}
        deriving (Show, Eq, Generic)

data Arra
      = Arra {fg :: String}
        deriving (Show, Eq, Generic)            
\end{lstlisting}

\section{Соответствие синтаксису JSON}

Типы данных, получаемые после работы программы полностью удовлетворяют синтаксису JSON и в полной мере покрывают все возможные случаи, в том числе поля-списки. Это говорит о том, что в языке \lstinline{Haskell} в полной мере могут использоваться так называемые «провайдеры типов» для \lstinline{JSON}-файлов.

\section{Развернутый пример}

Более развернутый пример см. в Приложении А.

\Conc

В данной работе была реализована генерация алгебраических типов данных функционального языка программирования \lstinline{Haskell} 
по исходным данным в формате \lstinline{JSON}. Данная идея заимствована из языка \lstinline{F#}, где были добавлены так называемые «поставщики типов» (\lstinline{Type Providers}). В тексте работы приводится обзор реализации задачи на \lstinline{Haskell}. При этом программа была опробована на нескольких вариантах входных данных.

При подготовке активно использовались сильные стороны функционального программирования. Выразительность и высокоуровневость языка \lstinline{Haskell} придает лаконичности коду, что позволяет зачастую решать сложные задачи с помощью достаточно простого и короткого кода. В качестве основных инструментов для реализации поставленной задачи используются библиотека \lstinline{Data.Aeson} и расширение \lstinline{Template Haskell}. Генерация производится с помощью рекурсивного прохода по абстрактному синтаксическому дереву и накопления результатов обхода в монаде \lstinline{State}.

Данная работа дает представление о средствах метапрограммирования на языке \lstinline{Haskell}, а также демонстрирует возможность применения идеи поставщиков типов в других языках программирования. 

% Печать списка литературы (библиографии)
\printbibliography[%{}
    heading=bibintoc%
    %,title=Библиография % если хочется это слово
]

\appendix
\addtocontents{toc}{%
    \protect\renewcommand{\protect\cftchappresnum}{\appendixname\space}%
    \protect\addtolength{\protect\cftchapnumwidth}{\widthof{\appendixname\space{}} - \widthof{Глава }}%
}%
\chapter{Развернутый пример}

\begin{ListingEnv}[H]
\begin{Verb}
{
  "glossary": {
    "title": "example glossary",
    "glossDiv": {
      "tit": "S",
      "glossList": {
        "glossEntry": {
          "id": "SGML",
          "sortAs": "SGML",
          "glossTerm": "Standard Generalized Markup Language",
          "acronym": "SGML",
          "abbrev": "ISO 8879:1986",
          "glossDef": {
            "para": "JS used to create markup languages such as DocBook.",
            "glossSeeAlso": [
              "GML",
              "XML"
            ]
          },
          "glossSee": "markup"
        }
      }
    }
  }
}
\end{Verb}
\caption{JSON-файл на входе программы}
\label{listing:inputGreatEx}
\end{ListingEnv}

\begin{ListingEnv}[H]
\begin{Verb}
data JSONData
      = JSONData {glossary :: Glossary}
      deriving (Generic, Show, Eq)
data Glossary
      = Glossary {title :: String, glossDiv :: GlossDiv}
      deriving (Generic, Show, Eq)
data GlossDiv
      = GlossDiv {tit :: String, glossList :: GlossList}
      deriving (Generic, Show, Eq)
data GlossList
      = GlossList {glossEntry :: GlossEntry}
      deriving (Generic, Show, Eq)
data GlossEntry
      = GlossEntry {glossDef :: GlossDef,
                    glossTerm :: String,
                    id :: String,
                    glossSee :: String,
                    abbrev :: String,
                    sortAs :: String,
                    acronym :: String}
      deriving (Generic, Show, Eq)
data GlossDef
      = GlossDef {glossSeeAlso :: [String], para :: String}
      deriving (Generic, Show, Eq)
\end{Verb}
\caption{Полученный тип данных}
\label{listing:outputGreatEx}
\end{ListingEnv}
% Файл со списком литературы: biblio.bib
% Подробно по оформлению библиографии:
% см. документацию к пакету biblatex-gost
% http://ctan.mirrorcatalogs.com/macros/latex/exptl/biblatex-contrib/biblatex-gost/doc/biblatex-gost.pdf
% и огромное количество примеров там же:
% http://mirror.macomnet.net/pub/CTAN/macros/latex/contrib/biblatex-contrib/biblatex-gost/doc/biblatex-gost-examples.pdf

\end{document}
