% В этом файле следует писать текст работы, разбивая его на
% разделы (section), подразделы (subsection) и, если нужно,
% главы (chapter).

% Предварительно следует указать необходимую информацию
% в файле SETUP.tex

%% В этот файл не предполагается вносить изменения

% В этом файле следует указать информацию о себе
% и выполняемой работе.

\documentclass [fontsize=14pt, paper=a4, pagesize, DIV=calc]%
{scrartcl}
% ВНИМАНИЕ! Для использования глав поменять
% scrartcl на scrreprt

% Здесь ничего не менять
\usepackage [T2A] {fontenc}   % Кириллица в PDF файле
\usepackage [utf8] {inputenc} % Кодировка текста: utf-8
\usepackage [russian] {babel} % Переносы, лигатуры

%%%%%%%%%%%%%%%%%%%%%%%%%%%%%%%%%%%%%%%%%%%%%%%%%%%%%%%%%%%%%%%%%%%%%%%%
% Создание макроса управления элементами, специфичными
% для вида работы (курс., бак., маг.)
% Здесь ничего не менять:
\usepackage{ifthen}
\newcounter{worktype}
\newcommand{\typeOfWork}[1]
{
	\setcounter{worktype}{#1}
}

% ВНИМАНИЕ!
% Укажите тип работы: 0 - курсовая, 1 - бак., 2 - маг.,
% 3 - бакалаврская с главами.
\typeOfWork{1}
% Считается, что курсовая и бак. бьются на разделы (section) и
% подразделы (subsection), а маг. — на главы (chapter), разделы и
%  подразделы. Если хочется,
% чтобы бак. была с главами (например, если она большая),
% надо выбрать опцию 3.

% Если при выборе 2 или 3 вы забудете поменять класс
% документа на scrreprt (см. выше, в самом начале),
% то получите ошибку:
% ./aux/appearance.tex:52: Package scrbase Error: unknown option ` chapterprefix=

%%%%%%%%%%%%%%%%%%%%%%%%%%%%%%%%%%%%%%%%%%%%%%%%%%%%%%%%%%%%%%%%%%%%%%%%
% Информация об авторе и работе для титульной страницы

\usepackage {titling}

% Имя автора в именительном падеже (для маг.)
\newcommand {\me}{%
И.\,И.~Иванов%
}

% Имя автора в родительном падеже (для курсовой и бак.)
\newcommand {\byme}{%
И.\,И.~Иванова%
}

% Любимый научный руководитель
\newcommand{\supervisor}%
{учёная степень, учёное звание /  должность И. О. Фамилия}

% идентифицируем пол (только для курсовой и бак.)
\newcommand{\bystudent}{
Студента %Студентки % Для курсовой: с большой буквы
}

% Год публикации
\date{2015}

% Название работы
\title{Потенциально длинное название работы\\на две или три строки}

% Кафедра
%
\newboolean{needchair}
\setboolean{needchair}{false} % на ФИИТ не пишется (false), на ПМИ есть (true)

\newcommand {\thechair} {%
Кафедра компьютерного и аналогового моделирования светлого будущего%
}

\newcommand {\direction} {%
Направление подготовки\\
Фундаментальная информатика и информационные технологии%
}% Прикладная математика и информатика

%%%%%%%%%%%%%%%%%%%%%%%%%%%%%%%%%%%%%%%%%%%%%%%%%%%%%%%%%%%%%%%%%%%%%%%%
% Другие настраиваемые элементы текста

% Листинги с исходным кодом программ: укажите язык программирования
\usepackage{listings}
\lstset{
    language=[ISO]C++,%  Язык указать здесь
    basicstyle=\small\ttfamily,
    breaklines=true,%
    showstringspaces=false%
    inputencoding=utf8x%
}
% полный список языков, поддерживаемых данным пакетом, есть,
% например, здесь (стр. 13):
% ftp://ftp.tex.ac.uk/tex-archive/macros/latex/contrib/listings/listings.pdf

% Нумерация списков: можно при необходимести
% изменять вид нумерации (например, добавлять правую скобку).
% По умолчанию буду списки вида:
% 1.
% 2.
% Изменять вид нумерации можно в начале нумерации:
% \begin{enumerate}[1)] (В квадратных скобках указан желаемый вид)
\usepackage[shortlabels]{enumitem}
                    \setlist[enumerate, 1]{1.}

% Гиперссылки: настройте внешний вид ссылок
\usepackage%
[pdftex,unicode,pdfborder={0 0 0},draft=false,%backref=page,
    hidelinks, % убрать, если хочется видеть ссылки: это
               % удобно в PDF файле, но не должно появиться на печати
    bookmarks=true,bookmarksnumbered=false,bookmarksopen=false]%
{hyperref}


\usepackage {amsmath}      % Больше математики
\usepackage {amssymb}
\usepackage {textcase}     % Преобразование к верхнему регистру
\usepackage {indentfirst}  % Красная строка первого абзаца в разделе

\usepackage {fancyvrb}     % Листинги: определяем своё окружение Verb
\DefineVerbatimEnvironment% с уменьшенным шрифтом
	{Verb}{Verbatim}
	{fontsize=\small}

% Вставка рисунков
\usepackage {graphicx}

% Общее оформление
% ----------------------------------------------------------------
% Настройка внешнего вида

%%% Шрифты

% если закомментировать всё — консервативная гарнитура Computer Modern
\usepackage{paratype} % профессиональные свободные шрифты
%\usepackage {droid}  % неплохие свободные шрифты от Google
%\usepackage{mathptmx}
%\usepackage {mmasym}
%\usepackage {psfonts}
%\usepackage{lmodern}
%var1: lh additions for bold concrete fonts
%\usepackage{lh-t2axccr}
%var2: the package below could be covered with fd-files
%\usepackage{lh-t2accr}
%\usepackage {pscyr}

% Геометрия текста

\usepackage{setspace}       % Межстрочный интервал
\onehalfspacing

\newlength\MyIndent
\setlength\MyIndent{1.25cm}
\setlength{\parindent}{\MyIndent} % Абзацный отступ
\frenchspacing            % Отключение лишних отступов после точек
\KOMAoptions{%
    DIV=calc,         % Пересчёт геометрии
    numbers=endperiod % точки после номеров разделов
}

                            % Консервативный вариант:
%\usepackage                % ручное задание геометрии
%[%                         % (не рекомендуется в проф. типографии)
%  margin = 2.5cm,
  %includefoot,
  %footskip = 1cm
%] %
%  {geometry}

%%% Заголовки

\ifthenelse{\equal{\theworktype}{2}}{%
\KOMAoptions{%
    numbers=endperiod,% точки после номеров разделов
    headings=normal,   % размеры заголовков поменьше стандартных
    chapterprefix=true,% Печатать слово Глава в магистерской
    appendixprefix=true% Печатать слово Приложение
}
}

% шрифт для оформления глав и названия содержания
\newcommand{\SuperFont}{\Large\sffamily\bfseries}

% Заголовок главы
\ifthenelse{\value{worktype} > 1}{%
\renewcommand{\SuperFont}{\Large\normalfont\sffamily}
\newcommand{\CentSuperFont}{\centering\SuperFont}
\usepackage{fncychap}
\ChNameVar{\SuperFont}
\ChNumVar{\CentSuperFont}
\ChTitleVar{\CentSuperFont}
\ChNameUpperCase
\ChTitleUpperCase
}

% Заголовок (под)раздела с абзацного отступа
\addtokomafont{sectioning}{\hspace{\MyIndent}}

\renewcommand*{\captionformat}{~---~}
\renewcommand*{\figureformat}{Рисунок~\thefigure}

% Плавающие листинги
\usepackage{float}
\floatstyle{ruled}
\floatname{ListingEnv}{Листинг}
\newfloat{ListingEnv}{htbp}{lol}[section]

% точка после номера листинга
\makeatletter
\renewcommand\floatc@ruled[2]{{\@fs@cfont #1.} #2\par}
\makeatother


%%% Оглавление
\usepackage{tocloft}

% шрифт и положение заголовка
\ifthenelse{\value{worktype} > 1}{%
\renewcommand{\cfttoctitlefont}{\hfil\SuperFont\MakeUppercase}
}{
\renewcommand{\cfttoctitlefont}{\hfil\SuperFont}
}

% слово Глава
\usepackage{calc}
\ifthenelse{\value{worktype} > 1}{%
\renewcommand{\cftchappresnum}{Глава }
\addtolength{\cftchapnumwidth}{\widthof{Глава }}
}

% Очищаем оформление названий старших элементов в оглавлении
\ifthenelse{\value{worktype} > 1}{%
\renewcommand{\cftchapfont}{}
\renewcommand{\cftchappagefont}{}
}{
\renewcommand{\cftsecfont}{}
\renewcommand{\cftsecpagefont}{}
}

% Точки после верхних элементов оглавления
\renewcommand{\cftsecdotsep}{\cftdotsep}
%\newcommand{\cftchapdotsep}{\cftdotsep}

\ifthenelse{\value{worktype} > 1}{%
    \renewcommand{\cftchapaftersnum}{.}
}{}
\renewcommand{\cftsecaftersnum}{.}
\renewcommand{\cftsubsecaftersnum}{.}
\renewcommand{\cftsubsubsecaftersnum}{.}

%%% Списки (enumitem)

\usepackage {enumitem}      % Списки с настройкой отступов
\setlist %
{ %
  leftmargin = \parindent, itemsep=.5ex, topsep=.4ex
} %

% По ГОСТу нумерация должны быть буквами: а, б...
%\makeatletter
%    \AddEnumerateCounter{\asbuk}{\@asbuk}{м)}
%\makeatother
%\renewcommand{\labelenumi}{\asbuk{enumi})}
%\renewcommand{\labelenumii}{\arabic{enumii})}

%%% Таблицы: выбрать более подходящие

\usepackage{booktabs} % считаются наиболее профессионально выполненными
%\usepackage{ltablex}
%\newcolumntype {L} {>{---}l}

%%% Библиография

\usepackage{csquotes}        % Оформление списка литературы
\usepackage[
  backend=biber,
  hyperref=auto,
  sorting=none, % сортировка в порядке встречаемости ссылок
  language=auto,
  citestyle=gost-numeric,
  bibstyle=gost-numeric
]{biblatex}
\addbibresource{biblio.bib} % Файл с лит.источниками

% Настройка величины отступа в списке
\ifthenelse{\value{worktype} < 2}{%
\defbibenvironment{bibliography}
  {\list
     {\printtext[labelnumberwidth]{%
    \printfield{prefixnumber}%
    \printfield{labelnumber}}}
     {\setlength{\labelwidth}{\labelnumberwidth}%
      \setlength{\leftmargin}{\labelwidth}%
      \setlength{\labelsep}{\dimexpr\MyIndent-\labelwidth\relax}% <----- default is \biblabelsep
      \addtolength{\leftmargin}{\labelsep}%
      \setlength{\itemsep}{\bibitemsep}%
      \setlength{\parsep}{\bibparsep}}%
      \renewcommand*{\makelabel}[1]{\hss##1}}
  {\endlist}
  {\item}
}{}

% ----------------------------------------------------------------
% Настройка переносов и разрывов страниц

\binoppenalty = 10000      % Запрет переносов строк в формулах
\relpenalty = 10000        %

\sloppy                    % Не выходить за границы бокса
%\tolerance = 400          % или более точно
\clubpenalty = 10000       % Запрет разрывов страниц после первой
\widowpenalty = 10000      % и перед предпоследней строкой абзаца

% ----------------------------


% Стили для окружений типа Определение, Теорема...
% Оформление теорем (ntheorem)

\usepackage [thmmarks, amsmath] {ntheorem}
\theorempreskipamount 0.6cm

\theoremstyle {plain} %
\theoremheaderfont {\normalfont \bfseries} %
\theorembodyfont {\slshape} %
\theoremsymbol {\ensuremath {_\Box}} %
\theoremseparator {:} %
\newtheorem {mystatement} {Утверждение} [section] %
\newtheorem {mylemma} {Лемма} [section] %
\newtheorem {mycorollary} {Следствие} [section] %

\theoremstyle {nonumberplain} %
\theoremseparator {.} %
\theoremsymbol {\ensuremath {_\diamondsuit}} %
\newtheorem {mydefinition} {Определение} %

\theoremstyle {plain} %
\theoremheaderfont {\normalfont \bfseries} 
\theorembodyfont {\normalfont} 
%\theoremsymbol {\ensuremath {_\Box}} %
\theoremseparator {.} %
\newtheorem {mytask} {Задача} [section]%
\renewcommand{\themytask}{\arabic{mytask}}

\theoremheaderfont {\scshape} %
\theorembodyfont {\upshape} %
\theoremstyle {nonumberplain} %
\theoremseparator {} %
\theoremsymbol {\rule {1ex} {1ex}} %
\newtheorem {myproof} {Доказательство} %

\theorembodyfont {\upshape} %
%\theoremindent 0.5cm
\theoremstyle {nonumberbreak} \theoremseparator {\\} %
\theoremsymbol {\ensuremath {\ast}} %
\newtheorem {myexample} {Пример} %
\newtheorem {myexamples} {Примеры} %

\theoremheaderfont {\itshape} %
\theorembodyfont {\upshape} %
\theoremstyle {nonumberplain} %
\theoremseparator {:} %
\theoremsymbol {\ensuremath {_\triangle}} %
\newtheorem {myremark} {Замечание} %
\theoremstyle {nonumberbreak} %
\newtheorem {myremarks} {Замечания} %


% Титульный лист
% Макросы настройки титульной страницы
% В этот файл не предполагается вносить изменения

%\usepackage {showframe}

% Вертикальные отступы на титульной странице
\newcommand{\vgap}{\vspace{16pt}}

% Помещение города и даты в нижний колонтитул
\usepackage{scrlayer}
\DeclareNewLayer[
  foot,
  foreground,
  contents={%
    \raisebox{\dp\strutbox}[\layerheight][0pt]{%
      \parbox[b]{\layerwidth}{\centering Ростов-на-Дону\\ \thedate%
       \\\mbox{}
       }}%
  }
]{titlepage.foot.fg}
\DeclareNewPageStyleByLayers{titlepage}{titlepage.foot.fg}


\AtBeginDocument %
{ %
  %
  \begin{titlepage}
  %
    \thispagestyle{titlepage}

    {\centering
    %
    \MakeTextUppercase {МИНИСТЕРСТВО ОБРАЗОВАНИЯ И НАУКИ РФ}

    \vgap

    Федеральное государственное автономное образовательное\\
    учреждение высшего образования\\
    \MakeTextUppercase {Южный федеральный университет}

    \vgap

	Институт математики, механики и компьютерных наук
    имени~И.\,И.\,Воровича

    \vgap

    \direction

    \ifthenelse{\boolean{needchair}}{
    \vgap

    \thechair}{}

    \vspace* {\fill}

    \ifthenelse{\value{worktype} = 2}{%
    \me

    \vgap}{}

    {\usefont{T2A}{PTSansCaption-TLF}{m}{n}
    \MakeTextUppercase{\thetitle}}

    \ifthenelse{\value{worktype} = 2}{%
     \vgap

    Магистерская диссертация}{}
    \ifthenelse{\value{worktype} = 0}{
     \vgap

    Курсовая работа
    }{}%
    \ifthenelse{\value{worktype} = 1 \OR \value{worktype} = 3}{
     \vgap

    Выпускная квалификационная работа\\
    на степень бакалавра
    }{}%

    \vspace {\fill}

    \begin{flushright}
    \ifthenelse{\value{worktype} = 0 \OR 
                \value{worktype} = 1 \OR
                \value{worktype} = 3}{
      \bystudent \ifthenelse{\value{worktype} = 0}{3}{4}\ курса\\
      \byme
    }{}

    \vgap

    Научный руководитель:\\
    \supervisor\\
    \ifthenelse{\value{worktype} = 2}{%
    Рецензент:\\
    ученая степень, ученое звание, должность
    И. О. Фамилия
    }{}
	\end{flushright}
    \ifthenelse{\value{worktype} = 0}{
    \vspace{\fill}
            \begin{flushleft}
              \begin{tabular}{cc}
                \underline{\hspace{4cm}}&\underline{\hspace{5cm}}\\
                {\small оценка (рейтинг)} & {\small  подпись руководителя}\\
              \end{tabular}
            \end{flushleft}
    }{}
  	\vspace {\fill}
  %Ростов-на-Дону

    %\thedate

  }\end{titlepage}
  %
  %
  \tableofcontents
  %
  \clearpage
} %



% Команды для использования в тексте работы


% макросы для начала введения и заключения
\newcommand{\Intro}{\addsec{Введение}}
\ifthenelse{\value{worktype} > 1}{%
    \renewcommand{\Intro}{\addchap{Введение}}%
}

\newcommand{\Conc}{\addsec{Заключение}}
\ifthenelse{\value{worktype} > 1}{%
    \renewcommand{\Conc}{\addchap{Заключение}}%
}

% Правильные значки для нестрогих неравенств и пустого множества
\renewcommand {\le} {\leqslant}
\renewcommand {\ge} {\geqslant}
\renewcommand {\emptyset} {\varnothing}

% N ажурное: натуральные числа
\newcommand {\N} {\ensuremath{\mathbb N}}

% значок С++ — используйте команду \cpp
\newcommand{\cpp}{%
C\nolinebreak\hspace{-.05em}%
\raisebox{.2ex}{+}\nolinebreak\hspace{-.10em}%
\raisebox{.2ex}{+}%
}

% Неразрывный дефис, который допускает перенос внутри слов,
% типа жёлто-синий: нужно писать жёлто"/синий.
\makeatletter
    \defineshorthand[russian]{"/}{\mbox{-}\bbl@allowhyphens}
\makeatother


\endinput

% Конец файла

\graphicspath{ {img/} }

\NewBibliographyString{langjapanese}
\NewBibliographyString{fromjapanese}

\begin{document}

\Intro
В данной работе была реализована генерация алгебраических типов данных языка программирования Haskell 
по исходным данным в формате JSON.

В качестве основных инструментов для реализации поставленной задачи используются 
библиотека Data.Aeson~\cite{aeson} и расширение Template Haskell~\cite{tempHaskell}. Генерация 
производится с помощью рекурсивного прохода по абстрактному синтаксическому дереву (далее - AST) и 
накопление результатов обхода в монаде State ~\cite{stateM}.

Ключевыми моментами в реализации программы являются:

\begin{itemize}
  \item Получение AST по исходному JSON
  \item Преобразование AST в структруру для генерации алгебраического типа  
  \item Вклейка ("splicing") сгенерированного типа данных в код
\end{itemize}

\chapter{Предварительные сведения}

\section{JavaScript Object Notation}

JSON (\lstinline{JavaScript Object Notation})~\cite{jsonStandart} - простой формат обмена данными, основанный на подмножестве языка программирования JavaScript. При этом независим и может использоваться практически любым языком программирования. Файл в формате JSON представляет неупорядоченное множество пар ключ-значение, значения которого могут состоят из:  

\begin{itemize}
  \item Объектов (выделяются \{ ... \})
  \item Массивов (выделяются [ ... ])
  \item Строк
  \item Логических выражений (true|false)
  \item null-значений
\end{itemize}

\begin{ListingEnv}[H]
Пример данных в формате JSON:
\begin{Verb}
{     
    "firstName": "John",
    "lastName" : "Smith",
    "age" : 25
}
\end{Verb}
\caption{Пример данных в формате JSON}
\label{listing:jsonExample}
\end{ListingEnv}

Структуру хорошо можно выразить благодаря схемам, представленным на рисунках ~\ref{fig:objectGr}, ~\ref{fig:arrayGr} и ~\ref{fig:valueGr}.

\begin{figure}[!ht]
\centering
\includegraphics[width=\textwidth]{object}
\caption{\label{fig:objectGr}Объект}
\end{figure}

\begin{figure}[!ht]
\centering
\includegraphics[width=\textwidth]{array}
\caption{\label{fig:arrayGr}Массив}
\end{figure}

\begin{figure}[!ht]
\centering
\includegraphics[width=\textwidth]{value}
\caption{\label{fig:valueGr}Значение}
\end{figure}

\section{Алгебраические типы данных}

Алгебраические типы данных (далее - АТД) - вид составных типов, представленных типом-произведением, типом-суммой, либо типом-суммой типов-произведений.~\cite{haskellGreatGood} Последний вариант хорошо илюстрируется примером двоичного дерева:

\begin{lstlisting}[language=Haskell]
data Tree a = Leaf a
          | Node (Tree a) (Tree a)
\end{lstlisting}

В примере \lstinline{Tree} является конструктором типа с одним типовым параметром. \lstinline{Leaf} и \lstinline{Node} - конструкторы значений. 

Также есть еще один вариант определения типа, называемый синтаксис записи с именованными полями:

\begin{lstlisting}[language=Haskell]
data Person = Person { firstName :: String
                       , lastName :: String
                       , age :: Int }
\end{lstlisting}

Вместо простого перечисления типов, мы создаем некую структуру и наполяем ее полями и их значениями. Главная выгода – такой синтаксис генерирует функции для извлечения полей. Также создается удобство чтения и понимая типа данных. Такие типы напоминают своей структурой \lstinline{JSON}-файлы. 

\section{Монада State}

Монада State применима, когда имеется некоторое состояние, которое мы постоянно изменяем. Cамое интересное, что при таком способе мы трансформируем состояние, но при этом не теряем "чистоту" функций. 

\subsection{Control.Monad.State}

Модуль Control.Monad.State~\cite{stateControl} определяет тип, оборачивающий вычисление с состоянием:

\begin{lstlisting}[language=Haskell]
newtype State s a = State {runState :: s -> (a,s) }
\end{lstlisting} 

Как видно из определения, вычисление в монаде State возвращает некоторый результат, при этом меняет состояние, при необходимости. Операции с состояними реализованы следующими функциями: \lstinline{get}(получает состояние) и \lstinline{put}(изменяет состояние на новое). В листинге~\ref{listing:stateGetPut} приведен простой пример, хорошо демонстрирующий возможности монады \lstinline{State}.

\begin{ListingEnv}[H]
\begin{Verb}
tick :: State Int Int
tick = do n <- get
	  put (n+1)
          return n

ghci> runState tick 3
(3,4)
\end{Verb}
\caption{Пример использования монады State}
\label{listing:stateGetPut}
\end{ListingEnv}

Также модуль содержит полезные функции для работы с состояниями. Самые важные представлены в листинге~\ref{listing:stateFunc}. Функции с постфиксом -State отличаются типом возвращаемого значения. Необходимое поведение можно выбрать исходя из сигнатуры.

\begin{ListingEnv}[H]
\begin{Verb}
modify :: MonadState s m => (s -> s) -> m ()

execState:: State s a -> s -> s

runState:: State s a -> s -> (a, s)

evalState:: State s a -> s -> a
\end{Verb}
\caption{Функции модуля Control.Monad.State}
\label{listing:stateFunc}
\end{ListingEnv}

Например, функция \lstinline{modify} преобразует внутреннее состояние функцией, которую получает на вход. Можно реализовать код листинга ~\ref{listing:stateGetPut} через \lstinline{modify} (листинг~\ref{listing:modifyState}).

\begin{ListingEnv}[H]
\begin{Verb}
tick :: State Int Int
tick = do modify (+1)
          return n

ghci> runState tick 3
(3,4)
\end{Verb}
\caption{Функция modify модуля Control.Monad.State}
\label{listing:modifyState}
\end{ListingEnv}
	
\section{Библиотека Data.Aeson}

\subsection{Общий способ}

Data.Aeson - библиотека для работы с файлами в формате JSON, написанная на языке Haskell. В библиотеке используются два основных класса типов - \lstinline{FromJSON} и \lstinline{ToJSON}.~\cite{aesonEx} Типы, имеющие возможность кодирования/декодирования, должны быть экземплярами классов FromJSON, ToJSON. Самый простой способ использования библиотеки - определить тип данных и экземпляры \lstinline{FromJSON}, \lstinline{ToJSON}. 

Существует возможность прописать экземпляры для кодирования/ декодирования по умолчанию, благодаря инструкции компилятора \lstinline{LANGUAGE} и экземпляру \lstinline{Generic}. Рассмотрим листинг~\ref{listing:genericData}, демонстрирующий данную возможность с условным типом данных 

\begin{ListingEnv}[H]
\begin{Verb}
instance ToJSON DataName

instance FromJSON DataName
\end{Verb}
\caption{Создание экземпляров по умолчанию}
\label{listing:genericData}
\end{ListingEnv}

Data.Aeson имеют свой собственный тип для представления конвертируемого JSON файла. Этот тип называется Value и имеет 6 конструкторов:

\begin{ListingEnv}[H]
\begin{Verb}
data Value
  = Object Object
  | Array Array
  | String Text
  | Number Scientific
  | Bool Bool
  | Null
\end{Verb}
\caption{Конструкторы Value}
\label{listing:value}
\end{ListingEnv}

\subsection{Работа с AST}

Aeson позволяет получить абстрактное синтаксическое дерево по JSON. Это бывает полезно, когда неизвестно какой тип данных 
соотвествует входному файлу. Имея AST, можно написать функцию для его обхода.~\cite{aesonEx}

В качестве примера рассмотрим получение AST для двух случаев: простой (листинг~\ref{listing:astGetSimple}) и более сложный со вложенными объектами (листинг~\ref{listing:astGetComp}).

\begin{ListingEnv}[H]
\begin{Verb}
decode :: FromJSON a => ByteString -> Maybe a

ghci> decode "{\"foo\": 123}" :: Maybe Value
Just (Object (fromList [("foo",Number 123)]))
\end{Verb}
\caption{JSON без вложенных объектов}
\label{listing:astGetSimple}
\end{ListingEnv}

\begin{ListingEnv}[H]
\begin{Verb}
ghci> decode "{\"foo\": [\"abc\",\"def\"]}" :: Maybe Value
Just (Object (fromList [("foo",Array (fromList [String "abc", 
                                               String "def"]))]))
\end{Verb}
\caption{JSON со вложенными объектами}
\label{listing:astGetComp}
\end{ListingEnv}

Помимо простого кодирования/декодирования JSON она также позволяет удобным образом писать сериализаторы и десериализаторы для произвольных типов.

\section{Template Haskell}

Template Haskell - это расширение языка Haskell, реализующее средства для метапрограммирования.~\cite{extensionHub} Оно позволяет использовать Haskell как язык управления и как управляемый язык. Так как работа ведется с расширением, то в исходник необходимо добавить директиву:

\begin{lstlisting}[language=Haskell]
{-# LANGUAGE TemplateHaskell #-}.
\end{lstlisting}

\subsection{Монада Q}

Монада \lstinline{Q} оборачивает Функция \lstinline{lift} поднимаеь

\subsection{Вклейка(splicing)}

Вклейка кода производится оператором  \$(...), который разворачивает шаблон с данным параметром в обычный Haskell код во время компиляции и вклеивает его на то же место. 

\subsection{Модуль Language.Haskell.TH.Syntax~\cite{coverHaskell}}

Тип \lstinline{Exp} определен в модуле Language.Haskell.TH.Syntax и представляет собой абстрактное синтаксическое дерево (AST) кода на Haskell.~\cite{thSyntax} \lstinline{Exp} имеет около 25 конструкторов значения. В листинге ~\ref{listing:expConstr} представлены только некоторые.  

\begin{ListingEnv}[H]
\begin{Verb}
data Exp
       = VarE Name
       | AppE Exp Exp
       | MultiIfE [(Guard, Exp)]
       | CondE Exp Exp Exp
       | ...
\end{Verb}
\caption{Конструкторы Exp}
\label{listing:expConstr}
\end{ListingEnv}

\chapter{Пример практического использования}

Git-репозиторий с исходным кодом на языке \lstinline{Haskell} доступен по адресу ~\cite{diploma}. 

\section{Простой пример}
Для начала будет рассмотрен простой пример. На вход программе подается простой \lstinline{JSON}, т.е. без вложенных объектов (листинг ~\ref{listing:json1}).

\begin{ListingEnv}[H]
\begin{Verb}
{
    "name" : "Joe",
    "age" : 25,
    "avg" : 4,
    "arr" : [1,2,3]
}
\end{Verb}
\label{listing:json1}
\end{ListingEnv}

В итоге мы получаем тип данных с именованными полями, полностью соответсвующий постановке задачи:

\begin{lstlisting}[language=Haskell]
data JSONData 
        = JSONData {arr :: [Float]
                    name :: String,
                    age :: Float,
                    avg :: Float}
          deriving (Show, Eq, Generic)
\end{lstlisting}

\section{Более сложный пример}

На вход программе подается \lstinline{JSON} со вложенным объектом. (Листинг ~\ref{listing:json2})  Интерес представляет то, как он будет представлен. 

\begin{ListingEnv}[H]
\begin{Verb}
{
    "name" : "Joe",
    "age" : 25,
    "avg" : 4,
    "arra" : 
             {
                 "fg" : "JSONTest"  
             }
}
\end{Verb}
\label{listing:json2}
\end{ListingEnv}

После выполнения программы мы получаем 2 типа данных, что полностью соответствует заявленным требованиям к программе. Первый тип в своем определении использует второй. 

\begin{lstlisting}[language=Haskell]
data JSONData
      = JSONData {name :: String,
                  arra :: Arra,
                  age :: Float,
                  avg :: Float}
        deriving (Show, Eq, Generic)

data Arra
      = Arra {fg :: String}
        deriving (Show, Eq, Generic)            
\end{lstlisting}




% Печать списка литературы (библиографии)
\printbibliography[%{}
    heading=bibintoc%
    %,title=Библиография % если хочется это слово
]
% Файл со списком литературы: biblio.bib
% Подробно по оформлению библиографии:
% см. документацию к пакету biblatex-gost
% http://ctan.mirrorcatalogs.com/macros/latex/exptl/biblatex-contrib/biblatex-gost/doc/biblatex-gost.pdf
% и огромное количество примеров там же:
% http://mirror.macomnet.net/pub/CTAN/macros/latex/contrib/biblatex-contrib/biblatex-gost/doc/biblatex-gost-examples.pdf

\end{document}
